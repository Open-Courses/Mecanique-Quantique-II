%        File: Meca.Quantique.II.tex
%     Created: Mit Feb 18 09:00  2015 C
% Last Change: Mit Feb 18 09:00  2015 C
%
\documentclass[a4paper, 12pt]{report}

\usepackage[french]{babel}
\usepackage[T1]{fontenc}
\usepackage[utf8]{inputenc}
\usepackage{hyperref}


\usepackage{amsmath}
\usepackage{amsthm}
\usepackage{amssymb}
\usepackage{physics}

\usepackage{amsmath}

\newcommand{\naturel}{\mathbb{N}}
\newcommand{\integer}{\mathbb{Z}}
\newcommand{\rational}{\mathbb{Q}}
\renewcommand{\real}{\mathbb{R}} %Already defined in physics package
\newcommand{\complex}{\mathbb{C}}

%Logical
\def\implies{\Rightarrow}
\def\equiv{\Leftrightarrow}

%Set theory
\def\union{\cup}
\def\inter{\cap}
\newcommand{\comp}[1]{#1^{c}}

%Topology
\def\interior{\mathring}
\def\adh{\overline}

\def\Ldeux{\mathcal{L}^{2}}
\def\Ldeuxstar{(\mathcal{L}^{2})^{*}}

%GSsequence :
%		#1 : represention of elements of the sequences
%		#2 : indices
%		#3 : set definition
\newcommand{\GSsequence}[3]{$(#1_{#2})_{#2 \in #3}$}

%GSset :
%		#1 : global set
%		#2 : condition
\newcommand{\GSset}[2]{$\left\{#1 \, | \, #2 \right\}$}

%GSprodSet :
%		#1 : indice
%		#2 : begin indice
%		#3 : end indice
%		#4 : set
\newcommand{\GSprodSet}[4]{$\displaystyle \prod_{#1 = #2}^{#3} #4_{#1}$}

%GSsum :
%		#1 : indice
%		#2 : begin indice
%		#3 : end indice
%		#4 : element
\newcommand{\GSsum}[4]{$\displaystyle \sum_{#1 = #2}^{#3}$ #4}

\newcommand{\GSintervalCC}[2]{$\left[#1, #2\right]$}

%Analysis :

%GSApplication :
%       #1 : name funtion
%       #2 : begin set
%       #3 : end set
\newcommand{\GSfunction}[3]{#1 : #2 $\rightarrow$ #3}
%GSnorme :
%		#1 : elements which norme is applied on
\newcommand{\GSnorme}[1]{$||#1||$}

%GSnormeDef :
%		#1 : elements which norme is applied on
%		#2 : norme indice
\newcommand{\GSnormeDef}[2]{$||#1||_{#2}$}

%GSnormedSpace :
%		#1 : vectorial space
%		#2 : \GSnorme[Def] with dot as element.
\newcommand{\GSnormedSpace}[2]{$($#1, #2$)$}

%GSdual
%		#1 : vectorial space
\newcommand{\GSdual}[1]{#1^{*}}

%GSbidual
%		#1 : vectorial space
\newcommand{\GSbidual}[1]{#1^{**}}

\newcommand{\GSunitBoule}[1]{\mathcal{B}_{#1}}

\newcommand{\GSweakTopo}[1]{$\sigma(#1, #1^{*})$}
\newcommand{\GSpreweakTopo}[1]{$\sigma(#1^{*}, #1)$}

%GSendomorphism
\newcommand{\GSendomorphism}[1]{End(#1)}

%GShomomorphisme
\newcommand{\GShomomorphisme}[2]{Hom(#1, #2)}

%GScontinueEndo
\newcommand{\GScontinueEndo}[1]{$\mathcal{L}(#1)$}

%\GScontinueHomo
\newcommand{\GScontinueHomo}[2]{$\mathcal{L}(#1, #2)$}

%Model theory

%Ultraproduct
%	1 : indice elements
%	2 : set which contains indices
%	3 : ultrafilter
%	4 : models represention
\newcommand{\GSultraproduct}[4]{$\displaystyle {\prod_{#1 \in #2}}^{#3}#4_{#1}$}

%Ultrapower
%	1 : indice elements
%	2 : set which contains indices
%	3 : ultrafilter
%	4 : model
\newcommand{\GSultrapower}[4]{$\displaystyle {\prod_{#1 \in #2}}^{#3}#4$}

%Substructures
\newcommand{\GSsubstructure}[2]{$#1 \subseteq #2$}

%Elementary Substructures.
\newcommand{\GSelemSubstructure}[2]{$#1 \preceq #2$}

%Elementary equivalent structures
\newcommand{\GSelemEquivStructure}[3]{$#2 \equiv_{#1} #3$}

\usepackage{amsfonts}
\usepackage{amssymb}
\usepackage{amsmath}
\usepackage{amsthm}
\usepackage{mathrsfs}

\newtheorem{definition}{Définition}[chapter]

\newtheorem{proposition}{Proposition}[chapter]
\newtheorem{lemma}[proposition]{Lemme}
\newtheorem{corollary}[proposition]{Corollaire}
\newtheorem{theorem}[proposition]{Théorème}

\newtheorem{exemple}{Exemple}[chapter]
\newtheorem*{question}{Questions}
\newtheorem*{remarque}{Remarque}

\newtheorem{exercice}{Exercice}[chapter]


\title{Mécanique quantique (expliqué aux matheux)}
\author{}

\begin{document}

\maketitle

\begin{abstract}
Ce cours se concentrera sur l'aspect mathématique de la mécanique quantique,
sans oublier les exemples physiques (atome d'hydrogène, oscillateur harmonique).
\end{abstract}

\tableofcontents

\chapter{Outils mathématiques de la mécanique quantique}

La mécanique quantique s'intérèsse essentiellement aux espaces de Hilbert, et
particulièrement de $\Ldeux(\real^{n}, \complex)$. Dans la suite de ce
cours, nous noterons $\Ldeux$ pour $\Ldeux(\real^{n}, \complex)$, et le corps de
base sera toujours $\complex$.
Le dual de $\mathcal{H}$, l'espace vectoriel des formes linéaires sur
$\mathcal{H}$ à valeur dans $\mathbb{K}$, est noté $\mathcal{H}^{*}$.

Le produit scalaire sur $\Ldeux$ est défini par l'intégrale: $\bra{f}\ket{g} =
\int_{\real^{n}} f(x)^{*} g(x) \mathrm{d}x$. La norme est alors $\norm{f}^{2} =
\int_{\real^{n}} \abs{f(x)} \mathrm{d}x$ où $\abs{.}$ est la norme usuelle sur
$\complex$.

\section{Notation de Dirac}

\begin{definition} [Ket]
	Chaque élément $\phi$ de $\Ldeux$ est appelé \textbf{'ket'}, et est noté
	$\ket{\phi}$.
\end{definition}

Comme $\Ldeux$ est un espace de Hilbert, un produit scalaire existe dessus. Nous
pouvons définir un élément de $\Ldeuxstar$ grace à chaque élément $\ket{\phi}$
de $\Ldeux$: le produit scalaire associé à $\ket{\phi}$.

\begin{definition} [Bra]
	Soit un ket $\ket{\phi}$. On définit \textbf{le bra associé au ket
	$\ket{\phi}$}, et on le note $\bra{\phi}$, par l'application $\phi^{*} :
	\Ldeux \rightarrow \complex : \ket{\psi} \rightarrow \bra{\phi}\ket{\psi} =
	\int_{\real^{n}} \phi(x)^{*} \psi(x) \mathrm{d}x$.
	On remarque que $\bra{\phi} \in \Ldeuxstar$, et n'est rien d'autre que le
	produit scalaire où la première composante est le ket $\ket{\phi}$.

	La notation de Dirac est motivée par pouvoir utiliser la notation du produit
	scalaire.
	En effet, quand on applique un ket $\ket{\psi}$ au bra $\bra{\phi}$,
	formellement, on écrit $\bra{\phi}(\ket{\psi}) = \bra{\phi}\ket{\psi}$.
\end{definition}

$\Ldeux$ étant un espace vectoriel, on peut réaliser des combinaisons linéaires
de deux kets. Etant donné une combili de deux kets $\ket{\phi}$ et $\ket{\psi}$
quel est le bra associé à $\alpha \ket{\phi} + \beta \ket{\psi}$ ?

Il faut se souvenir que les propriétés du produit scalaire implique les
propositions suivantes.

\begin{enumerate}
	\item Le bra associé à $\alpha \ket{\phi} + \beta \ket{\psi}$ est $\alpha^{*}
	\bra{\phi} + \beta^{*} \bra{\psi}$ (anti linéarité).
	\item $\bra{\phi}$ est linéaire.
	\item $\bra{\phi}\ket{\psi} = \bra{\psi}\ket{\phi}^{*}$.
	\item $\braket{\phi} = 0$ ssi $\ket{\phi} = 0$.
	\item Si $\ket{\phi} = 0_{\Ldeux}$ alors $\bra{\phi} = 0_{\Ldeuxstar}$.
	\item $|\ket{\phi}|^{2} = \braket{\phi} \braket{\phi}^{*}$ (la norme dérive
		du produit scalaire car on est dans un espace de Hilbert).
\end{enumerate}

Attention, toutes éléments de $\Ldeuxstar$ n'est pas construit sur un produit
scalaire! Donc tout ket est associé à un bra (avec le produit scalaire), mais
tout bra n'est pas associé à un ket (toute forme linéaire n'est pas issue d'un
produit scalaire).

\section{Opérateurs sur $\Ldeux$: $\GSendomorphism{\Ldeux}$}

\begin{definition} [Opérateur]
	\textbf{Un opérateur $A$} est un élément de $\GSendomorphism{\Ldeux}$,
c'est-à-dire une application linéaire $A : \Ldeux \rightarrow \Ldeux :
\ket{\phi} \rightarrow A(\ket{\phi})$. En particulier, pour $\alpha, \beta \in
\complex$, $\ket{\phi}, \ket{\psi} \in \Ldeux$, $A(\alpha \ket{\phi} + \beta
\ket{\psi}) = \alpha A(\ket{\phi} + \beta A(\ket{\psi})$.
\end{definition}

$\Ldeux$ est une algèbre de Banach pour le produit de deux fonctions de
$\Ldeux$.
$\GSendomorphism{\Ldeux}$ est en particulier une algèbre de Banach avec
l'opération produit, c'est-à-dire à deux fonctions $A, B \in
\GSendomorphism{\Ldeux}$, on définit $AB \in \GSendomorphism{\Ldeux}$ tel
que $AB(\ket{\phi})= A(\ket{\phi})B(\ket{\phi})$, où $A(\ket{\phi})
B(\ket{\psi})$ est la multiplication de deux éléments de $\Ldeux$.
Cette opération n'est pas nécéssairement commutative, et elle motive la
définition suivante.

\begin{definition} [Commutateur]
	Soit $A, B \in \GSendomorphism{\Ldeux}$. On définit \textbf{le commutateur
	de $A$ et $B$}, par l'application linéaire $\comm{A}{B} \in \Ldeux$ tel que
	$\comm{A}{B} = AB - BA$ ($\forall x \in \real^{n}$, $\comm{A}{B}(x) =
	A(x)B(x) - B(x)A(x)$).

	On dit que \textbf{$A$ et $B$ commute} si $\comm{A}{B} = 0_{\Ldeux}$,
	c'est-à-dire $AB = BA$.
\end{definition}

\begin{exemple}
	Soit $\ket{\phi}$ non nul. On souhaite réaliser une opération qui envoie
	chaque ket sur le sous-espace vectoriel engendré par $\ket{\phi}$. On
	utilise pour cela \textbf{l'opérateur de projection}:

	$P_{\phi} : \Ldeux \rightarrow <\phi> : \ket{\psi} \rightarrow
	\bra{\phi}\ket{\psi} \ket{\phi}$.
\end{exemple}

%\section{Opérateurs sur les bras: $\GSendomorphism{\Ldeuxstar}$}

%\begin{definition}
	%Soit un bra $\bra{\phi}$, et un opérateur $A \in \GSendomorphism{\Ldeux}$.
	%On définit l'opérateur $\bra{\phi} A \in \GSendomorphism{\Ldeux}$ tel que
	%$(\bra{\phi} A) (\ket{\psi}) = \bra{\phi}(A(\ket{\psi})) =
	%\bra{\phi}A\ket{\psi}$.
%\end{definition}

%$\bra{\phi} A$ est en fait la composition de $\bra{\phi} \in \Ldeuxstar$
%avec $A \in \GSendomorphism{\Ldeux}$.
%L'application composée est linéaire, et donc $\bra{\phi} (\alpha A + \beta B) =
%\alpha \bra{\phi} A + \beta \bra{\psi} B$.

\section{Opérateur adjoint}

Prenons un opérateur $A \in \GSendomorphism{\Ldeux}$. On souhaite définir un
opérateur sur $\GSendomorphism{\Ldeuxstar}$ qui a un lien direct avec $A$. Elle
nous amène à définir l'opérateur adjoint de $A$, qu'on note $A^{\dagger}$.

\begin{definition}
	Soit $A \in \GSendomorphism{\Ldeux}$. On définit \textbf{l'opérateur adjoint
	de $A$} comme l'application $A^{\dagger} : \Ldeuxstar \rightarrow \Ldeuxstar :
	\ket{\phi} \rightarrow A^{\dagger}(\bra{\phi}) = \bra{\phi} \circ A =
	\bra{\phi} A$.
\end{definition}

\begin{definition}
	Soit $A \in \GSendomorphism{\Ldeux}$. On dit que \textbf{$A$ est
	hermitique} si $A = A^{\dagger}$, c'est-à-dire quand $A$ et son adjoint
	coïncident.
\end{definition}

\section{Représentation}

\begin{definition}
	On appelle \textbf{représentation} une base de l'espace vectoriel étudié.
\end{definition}

Dans notre cas, on parlera \textbf{d'espace des états}, et \textbf{de
représentation de l'espace des états}.

\section{Equations aux valeurs propres, observables}

La mécanique quantique est très reliée à la théorie spectrale, c'est-à-dire
l'étude des valeurs propres et des vecteurs propres d'un opérateur (d'un
endomorphisme).

Rappelons d'abord quelques notions de théorie spectrale. Nous n'étudierons qu'un
particulier de la théorie spectrale. En effet, son domaine d'application est les
algèbres. Dans notre cas, nous étudierons les espaces d'endormorphismes qui sont
un cas particulier d'algèbre. Les notations seront plus aisées.

\begin{definition} [Valeur propre]
	Soit $A \in \GSendomorphism{\Ldeux}$, et $\lambda \in \complex$. On dit que
	\textbf{$\lambda$ est valeur propre de $A$} s'il existe $\ket{\phi} \in
	\Ldeux$ tel que $A \ket{\phi} = \lambda \ket{\phi}$.

	De manière équivalente, $\lambda$ est valeur propre si $\ker(A - \lambda
	Id_{\GSendomorphism{\Ldeux}}) \neq \emptyset$, c'est-à-dire que $A - \lambda
	Id_{\GSendomorphism{\Ldeux}}$ n'est pas injective.
\end{definition}

\begin{definition} [Spectre d'un opérateur]
	Soit $A \in \GSendomorphism{\Ldeux}$. \textbf{Le spectre de $A$} est
	l'ensemble de ses valeurs propres.
\end{definition}

\begin{definition} [Vecteur propre]
	Soit $A \in \GSendomorphism{\Ldeux}$, et $\ket{\phi} \in \Ldeux$ non nul. On
	dit que \textbf{$\ket{\phi}$ est vecteur propre pour la valeur propre
	$\lambda$} s'il existe une valeur propre $\lambda \in \complex$ tel que $A
	\ket{\phi} = \lambda \ket{\phi}$.

	De manière équivalente, on a $\ket{\phi} \in \ker(A - \lambda
	Id_{\GSendomorphism{\Ldeux}})$. Le noyau est donc au moins de dimension $1$.
\end{definition}

Le noyau de $A - \lambda Id_{\GSendomorphism{\Ldeux}}$ n'est pas nécéssairement
de dimension $1$, c'est-à-dire qu'il n'y a pas qu'un seul vecteur propre qui
engendre l'espace. On aura besoin dans la suite de faire la différence entre ces
deux cas, ce qui motive la définition suivante.

\begin{definition} [Dégénérescence]
	Une valeur propre $\lambda$ pour $A$ est dite \textbf{dégénérée} si $\ker(A
	- \lambda Id_{\GSendomorphism{\Ldeuxstar}})$ est au moins de dimension
	$2$.

	Sinon, si $\ker(A - \lambda Id_{\GSendomorphism{\Ldeuxstar}})$ est de
	dimension $1$, c'est-à-dire générée par un seul vecteur propre, on dit que
	$\lambda$ est \textbf{non-dégénérée}.
\end{definition}

\begin{definition} [Sous espace propre]
	\textbf{Le sous espace propre de $\lambda$ par rapport à $A$} est le noyau
	de $A - Id_{\GSendomorphism{\Ldeuxstar}}$.
	Sa dimension est notée $g$.
\end{definition}

\chapter{Les postulats de la mécanique quantique}
Page 212 - Chapitre III

\chapter{Application des postulats à des cas simples : spin 1/2 et système à
deux niveaux}
\label{chapter:application_spin_demi_systeme_deux_niveaux}
Page 384 - Chapitre IV

\chapter{Matrices de Pauli}
\label{chapter:matrice_pauli}
Page 415

\chapter{Etude de la molécule d'amoniac au moyen d'un modèle simple}

\chapter{Opérateur densité}

\chapter{Matrice densité d'un spin 1/2}

\chapter{Oscillateur harmonique à une dimension}

\chapter{Etude des états stationnaires en représentation $\GSset{\ket{x}}$. Polynomes d'Hermite}
\label{chapter:polynome_hermite}
Page 529

\chapter{Théorie de la perturbation}

\chapter{Résolution de l'oscillateur harmonique grace à la théorie de la
perturbation}

\chapter{Moment cinétique}
\label{chapter:moment_cinetique_rotations}
Page 697

\chapter{Propriété générale du moment cinétique}
\label{chapter:prop_generale_moment_cinetique}
Page 646 - Chapitre VI

\chapter{Particule dans un potentiel central}

\chapter{Etude de l'atome d'hydrogène}

\chapter{Oscillateur harmonique à trois dimensions}


\end{document}


